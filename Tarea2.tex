\documentclass{article}
\usepackage[utf8]{inputenc}
\usepackage[spanish]{babel}
\usepackage{graphicx}
\usepackage{multicol}
\usepackage{caption}
\usepackage{vmargin}
\usepackage[hidelinks]{hyperref}
\usepackage{physics}
\usepackage{amsfonts} 
\usepackage{commath}

%\usepackage{draftwatermark}
%\SetWatermarkText{\textsc{Borrador}} % por defecto Draft 
%\SetWatermarkScale{5} % para que cubra toda la página
%\SetWatermarkColor[rgb]{0.7,0.75,0.71} % por defecto gris claro
%\SetWatermarkAngle{55} % respecto a la horizontal

%\title{Tarea 1}
%\author{Raymundo Santana Carrillo\\ Computación cuántica \\ GUOHUA SUN}
%\date{}



\begin{document}
%\begin{multicols}{2}

\begin{center}
{\Large Quantum Computing \hspace{0.5cm} \\Homework 2}\\
\textbf{Raymundo Santana Carrillo}\\ %You should put your name here
Date: 6/02/2020 %You should write the date here.
\end{center}


   


\subsection*{Exercise chapter 2 Qbits and quantum states}

\begin{enumerate}
\item A quantum system is in the state

\[\frac{(1-i)}{\sqrt{3}}\ket{0}+ \frac{(1)}{\sqrt{3}}\ket{1}\]

if a measurment is made, what is the probability the system is in state or in state  ?



Solution.

\[  | \frac{(1-i)}{\sqrt{3}} |^2  = \frac{(1+i)}{\sqrt{3}}  \frac{(1-i)}{\sqrt{3}} = \frac{(1-i+i-i^2)}{3} = \frac{2}{3}  \]

\[  | \frac{(1)}{\sqrt{3}} |^2  = \frac{1}{3} \] 

\[\frac{2}{3} + \frac{1}{3} =1 \]
\item Two quantum states are given by

\[\ket{a}= \mqty( -4i \\ 2 ) , \ket{b}= \mqty( 1 \\ -1+i )\] 

\begin{enumerate}
   \item Find $\ket{a+b}$
   \item Calculate $3\ket{a}-2\ket{b}$
   \item Normalize $\ket{a},\ket{b}$
\end{enumerate}   

\[\ket{a+b} =  \mqty( -4i \\ 2 ) + \mqty( 1 \\ -1+i )= \mqty( 1-4i \\ 1+i )\] 

\[3\ket{a}-2\ket{b}=\frac{-2-12i}{8-2i} \]

\[ |a| =\sqrt{\bra{a}\ket{a}} = \mqty( 4i & 2)\mqty( -4i \\ 2 ) = -16i^2 +4=\sqrt{20} \]

\[ |b| =\sqrt{\bra{b}\ket{b}} = \mqty( 1 & -1-i)\mqty( 1 \\ -1+i ) =1+(-1-i)(-1-i)=\sqrt{3} \]

\[\ket{a} = \frac{\ket{a}}{|a|} = \frac{\mqty( -4i \\ 2 )}{\sqrt{20}}\]

\[\ket{b} = \frac{\ket{b}}{|b|} =  \frac{\mqty( 1 \\ -1+i )}{\sqrt{3}}\]


\item Another basis for $\mathbb{C}^{2}$  is
\[  \ket{+}=\frac{\ket{0}+\ket{1}}{\sqrt{2}}, \ket{-}=\frac{\ket{0}-\ket{1}}{\sqrt{2}}  \]

Invert this relation to express ${\ket{0} , \ket{1} }$ in terms of ${\ket{+} , \ket{-} }$.

\item A quantum system is in the state
\[ \ket{\psi}=\frac{3i\ket{0}+4\ket{1}}{\sqrt{5}} \] 

\begin{enumerate}
\item Is the state normalized?
\item Express the state in the $\ket{+}, \ket{-}$ basis.
\end{enumerate}




\item Use the Gram-Schmidt process to find an orthonormal basis for a subspace of
the four-dimensional space $\mathbb{R}^{4}$ spanned by

\[ \ket{u_{1}}=
\begin{pmatrix}
1\\
1\\
1\\
1\\
\end{pmatrix},  
\ket{u_{2}}=
\begin{pmatrix}
1\\
2\\
4\\
5\\
\end{pmatrix},
\ket{u_{3}}=
\begin{pmatrix}
1\\
-3\\
-4\\
-2\\
\end{pmatrix}
 \]

\item Photon horizontal and vertical polarization states are written as $\ket{h}$ and $\ket{v}$ ,
respectively. Suppose

\[ \ket{\psi_{1}}=\frac{1}{2}\ket{h}+\frac{\sqrt{3}}{2}\ket{v} \] 
\[ \ket{\psi_{2}}=\frac{1}{2}\ket{h}-\frac{\sqrt{3}}{2}\ket{v} \] 
\[ \ket{\psi_{3}}=\ket{h} \] 
\end{enumerate}









%\end{multicols}
\end{document}